\documentclass{article}
%----------------------------------------------------------------------------------------
%	Param setting
%----------------------------------------------------------------------------------------
\setlength{\parindent}{0cm}
\setlength{\parskip}{1em}
\setlength{\itemsep}{0.5em}
\newcommand{\killinitspace}{-0.7em}

\usepackage{geometry}
\geometry{a4paper, margin={0.75in, 1in}}

\usepackage{enumitem}
\setitemize{itemsep=0em, topsep=0em, leftmargin=1.5em, label=$\diamond$}

\usepackage{titlesec}
\titleformat{\section}       % Customise the \section command 
  {\Large\scshape\raggedright} % Make the \section headers large (\Large),
  {}{0em}
  {}                 % Can be used to insert code before the heading
  [\titlerule \vspace{-0.7em}]       % Inserts a horizontal line after the heading

\begin{document}
%----------------------------------------------------------------------------------------
%	Personal info
%----------------------------------------------------------------------------------------
{\Huge Yongho Shin}
\begin{itemize}[itemsep=-3pt, leftmargin=2.5pt, label=]
\item Postdoctoral researcher
\item Institute of Computer Science, University of Wrocław
\item ul. Joliot-Curie 15, 50-383 Wrocław,  Poland
\item Email: \texttt{yongho@cs.uni.wroc.pl}
\item Homepage: \texttt{https://yonghoshin36.github.io}
\end{itemize}

%----------------------------------------------------------------------------------------
%	Research interests
%----------------------------------------------------------------------------------------
\section{Research Interests}
Online/approximation algorithms for combinatorial optimization problems

%----------------------------------------------------------------------------------------
%	Education
%----------------------------------------------------------------------------------------
\section{Education}
\textbf{Ph.D.~in Computer Science, Yonsei University, South Korea} \hfill Mar.~2018 -- Aug.~2024
\vspace{\killinitspace}
\begin{itemize}
\item Dissertation topic: Relaxing hard requirements of online optimization via learning augmentation and limited revocability
\item Advisor: Hyung-Chan An
\end{itemize}

\textbf{B.S.~in Computer Science, Yonsei University, South Korea} \hfill Mar.~2012 -- Feb.~2018
\vspace{\killinitspace}
\begin{itemize}
\item Awarded \textsl{high honors at graduation}
\end{itemize}

%----------------------------------------------------------------------------------------
%	Education
%----------------------------------------------------------------------------------------
\section{Employment}
\textbf{Institute of Computer Science, University of Wrocław, Poland} \hfill Nov.~2024 -- present
\vspace{\killinitspace}
\begin{itemize}
\item Position: Postdoctoral researcher
\item Advisor: Jarosław Byrka
\end{itemize}

%----------------------------------------------------------------------------------------
%	Research papers
%----------------------------------------------------------------------------------------
\section{Research Papers}
\textbf{Yongho Shin}, Changyeol Lee, and Hyung-Chan An. On optimal consistency-robustness trade-off for learning-augmented multi-option ski rental. \emph{arXiv preprint arXiv:2312.02547}, 2023.
%\vspace{\killinitspace}
%\begin{itemize}
%\item This paper bridges the gap for both deterministic and randomized learning-augmented algorithms for the multi-option ski rental problem. For deterministic algorithms, we present a best-possible algorithm that matches the known lower bound; for randomized algorithms, we show the first nontrivial lower bound on the consistency-robustness trade-off, and also present an improved randomized algorithm. Our algorithm matches our lower bound on robustness within a factor of e/2 when the consistency is at most 1.086.
%\end{itemize}

\textbf{Yongho Shin}, Changyeol Lee, Gukryeol Lee, and Hyung-Chan An. Improved learning-augmented algorithms for the multi-option ski rental problem via best-possible competitive analysis. In \emph{Proceedings of the 40th International Conference on Machine Learning (ICML 2023)}, PMLR 202:31539-31561, 2023.
%\vspace{\killinitspace}
%\begin{itemize}
%\item Ski rental problems are one of the canonical problems in the field of online optimization. However, only deterministic algorithms were previously known for multi-option ski rental, with or without learning augmentation. We present the first randomized learning-augmented algorithm for this problem, surpassing previous performance guarantees given by deterministic algorithms. Our learning-augmented algorithm is based on a new, provably best-possible randomized competitive algorithm for the problem. Our results are further complemented by lower bounds for deterministic and randomized algorithms.
%\end{itemize}

Kangsan Kim, \textbf{Yongho Shin}, and Hyung-Chan An. Constant-factor approximation algorithms for parity-constrained facility location and $k$-center. \emph{Algorithmica 85}, 1883–1911, 2023.
%\vspace{\killinitspace}
%\begin{itemize}
%\item \emph{Facility location} is a prominent optimization problem in combinatorial optimization, and has been investigated under various settings. However, little is known on how the problem behaves in conjunction with parity constraints. This shortfall of understanding was rather disturbing when we consider the central role of \emph{parity} in the field of combinatorics. In this paper, we present the first constant-factor approximation algorithm for the facility location with parity constraints.
%\end{itemize}

\textbf{Yongho Shin} and Hyung-Chan An. Making three out of two: Three-way online correlated selection. In \emph{Proceedings of the 32nd International Symposium on Algorithms and Computation (ISAAC 2021)}, 49:1-49:17, 2021.
%\vspace{\killinitspace}
%\begin{itemize}
%\item \emph{Two-way online correlated selection (two-way OCS)} is an online algorithm that, at each timestep, takes a pair of elements from the ground set and irrevocably chooses one of the two elements, while ensuring negative correlation in the algorithm's choices. Fahrbach, Huang, Tao, and Zadimoghaddam initially invented OCS to tackle the edge-weighted online bipartite matching, and posed an open question: Can we obtain a nontrivial \emph{$n$-way OCS} for $n>2$? In this paper, we affirmatively answer this open question by presenting a \emph{three-way OCS}. We also show that our OCS yields a 0.5093-competitive algorithm for the edge-weighted online matching, demonstrating its usefulness.
%\end{itemize}

Kangsan Kim, \textbf{Yongho Shin}, and Hyung-Chan An. Constant-factor approximation algorithms for the parity-constrained facility location problem. In \emph{Proceedings of the 31st International Symposium on Algorithms and Computation (ISAAC 2020)}, 21:1-21:17, 2020.

\textbf{Yongho Shin}, Kangsan Kim, Seungmin Lee, and Hyung-Chan An. Online graph matching problem with a worst-case reassignment budget. \emph{arXiv preprint arXiv:2003.05175}, 2020.
%\vspace{\killinitspace}
%\begin{itemize}
%\item We propose to consider how requiring a \emph{worst-case} hard budget on the number of reassignments affects the algorithms' performances under various models of online graph matching. Through a simple algorithm exploiting a shortest augmenting path of length within the given budget, we demonstrate that even a small hard budget can yield significant performance advantage, compared to those algorithms that do not perform reassignments. Moreover, we further show that this algorithm is a best-possible deterministic algorithm for all those models.
%\end{itemize}

%----------------------------------------------------------------------------------------
%	Awards
%----------------------------------------------------------------------------------------
\section{Awards}
\textsl{High honors at graduation}, Yonsei University \hfill Feb.~2018

%----------------------------------------------------------------------------------------
%	Presentations
%----------------------------------------------------------------------------------------
\section{Talks and Presentations}
\textsl{Improved learning-augmented algorithms for the multi-option ski rental problem via best-possible competitive analysis}
\vspace{\killinitspace}
\begin{itemize}
\item Poster presentation at ICML 2023, Honolulu, HI, USA \hfill July~2023
\end{itemize}

\textsl{Making three out of two: Three-way online correlated selection}
\vspace{\killinitspace}
\begin{itemize}
\item Discrete Analysis Seminar, Yonsei University, Seoul, South Korea \hfill June~2024
\item Discrete Math Seminar, IBS DIMAG, Daejeon, South Korea \hfill May~2024
\item Theory Tea, Cornell University, Ithaca, NY, USA \hfill Dec.~2022
\item Presentation at ISAAC 2021, Fukuoka, Japan (virtual) \hfill Dec.~2021
\item Presentation at AAAC 2021, Tainan, Taiwan (virtual) \hfill Oct.~2021
\end{itemize}

\textsl{Constant-factor approximation algorithms for the parity-constrained facility location problem}
\vspace{\killinitspace}
\begin{itemize}
\item Presentation at ISAAC 2020, Hong Kong, China (virtual) \hfill Dec.~2020
\end{itemize}



%----------------------------------------------------------------------------------------
%	Research experience
%----------------------------------------------------------------------------------------
\section{Research Experience}
\textsl{Intern}, Cornell University \hfill Sept.~2022 -- Dec.~2022
\vspace{\killinitspace}
\begin{itemize}
\item Director: David B. Shmoys
\end{itemize}

\textsl{Undergraduate intern}, Yonsei University \hfill Jan.~2017 -- Feb.~2018
\vspace{\killinitspace}
\begin{itemize}
\item Advisor: Hyung-Chan An
\end{itemize}

%----------------------------------------------------------------------------------------
%	Teaching experience
%----------------------------------------------------------------------------------------
\section{Teaching Experience}
\textsl{Teaching assistant}, Yonsei University
\vspace{\killinitspace}
\begin{itemize}
\item CSI2103/CCO2103 Data Structures \hfill Spring 2018 -- 2021, 2023, 2024
\item CSI3108 Algorithm Analysis \hfill Fall 2018 -- 2021, 2023
\item AIC2130 Computer Algorithms for AI Applications \hfill Fall 2023
\item GEK6205 Design and Analysis of Optimization Algorithms \hfill Fall 2023
\end{itemize}

\textsl{Undergraduate voluntary tutor}, Yonsei University
\vspace{\killinitspace}
\begin{itemize}
\item CSI3108 Algorithm Analysis \hfill Fall 2016, 2017
\item CSI2103 Data Structures \hfill Spring 2017
\end{itemize}


%----------------------------------------------------------------------------------------
%	Misc experience
%----------------------------------------------------------------------------------------
\section{Miscellaneous Experience}
\textsl{Co-organizer} of Yonsei CS theory student group, Yonsei University \hfill Jan.~2023 -- Feb.~2024
\vspace{\killinitspace}
\begin{itemize}
\item Initiated a reading group of TCS students in and out of Yonsei University
\item Organizing seminar talks on various topics including mechanism design and quantum computing
\end{itemize}

\textsl{Web programmer}, Republic of Korea Air Force \hfill Nov.~2013 -- Aug.~2015
\vspace{\killinitspace}
\begin{itemize}
\item In fulfillment of mandatory military service
\end{itemize}
\end{document}